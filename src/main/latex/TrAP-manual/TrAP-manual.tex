\documentclass[a4paper,12pt]{amsart}
\usepackage[utf8x]{inputenc}
\usepackage{amsmath}
\usepackage[margin=2cm]{geometry}


% Title Page
\title{TrAP manual}
\author{Miroslav Bacak \and Vojtech Juranek}
\date{\today}
\begin{document}
\maketitle

%%%%%%%%%%%%%%%%%%%%%%%%%%%%%%%%%%%%%%%%%%%%%%%%%%%%%%%%%%%%%%%%%%%%%%%%%%%%%%%%%%%%%%%%%%%%%%%%%%%%%%%%%%%%%%%%%%%%%%%%%%%%%%%%%

%%%%%%%%%%%%%%%%%%%%%%%%%%%%%%%%%%%%%%%%%%%%%%%%%%%%%%%%%%%%%%%%%%%%%%%%%%%%%%%%%%%%%%%%%%%%%%%%%%%%%%%%%%%%%%%%%%%%%%%%%%%%%%%%%
\section{Introduction}

\textbf{TrAP} is a \textbf{Tr}ee \textbf{A}veraging \textbf{P}rogram computing medians and means of phylogenetic trees. It implements algorithms developed in~\cite{mm}. The program also allows to compute distance between two phylogenetic trees with the Owen-Provan algorithm~\cite{owenprovan}. You hence use this program for the following three tasks:
\begin{enumerate}
 \item to compute distance between two trees (option \texttt{-d}),
 \item to compute median of a set of trees (option \texttt{--median}),
 \item to compute mean of a set of trees (option \texttt{--mean}).
\end{enumerate}

%%%%%%%%%%%%%%%%%%%%%%%%%%%%%%%%%%%%%%%%%%%%%%%%%%%%%%%%%%%%%%%%%%%%%%%%%%%%%%%%%%%%%%%%%%%%%%%%%%%%%%%%%%%%%%%%%%%%%%%%%%%%%%%%%

%%%%%%%%%%%%%%%%%%%%%%%%%%%%%%%%%%%%%%%%%%%%%%%%%%%%%%%%%%%%%%%%%%%%%%%%%%%%%%%%%%%%%%%%%%%%%%%%%%%%%%%%%%%%%%%%%%%%%%%%%%%%%%%%%
\section{Command line usage}
Go to the directory containing \texttt{trap.jar} and type:
\begin{verbatim}
 java -jar trap.jar [options]
\end{verbatim}

\vspace{12pt}
Options:

\begin{tabular}{ll}
\\
\texttt{-d} & computes distance \\ 
\texttt{--median} & computes median \\ 
\texttt{--mean} & computes mean \\ 
\texttt{-c} & use cyclic version \\ 
\texttt{-n} & number of iterations \\ 
\texttt{-i <inputFile>} & specifies the input file \\ 
\texttt{-o <outputFile>} & specifies the input file \\ 
\texttt{-l <logFile>} & creates log file \\ 
\end{tabular}

\vspace{12pt}

The first tree options determine what you want to compute and are mutually exclusive. 

\subsection{Computing distances}
If you compute distance (i.e. choose option \texttt{-d}), then you have to specify the input file by the option \texttt{-i <inputFile>} and you can specify the output file by the option \texttt{-o <outputFile>}. No other options are applicable when computing distances. Your \texttt{<inputFile>} must be in the format described in Section~\ref{sec:input}. If it contains more than two trees, only first two are considered and distance between them is computed. The output is displayed on the screen and stored in \texttt{<outputFile>} provided you choose the \texttt{-o <outputFile>} option.

\subsection{Computing medians}
To compute a median, choose the \texttt{--median} option. Then you may require the cyclic-order version of the algorithm by selecting \texttt{-c}, see Section~\ref{sec:cycvran}. The number of iterations is a mandatory parameter, so type e.g. \texttt{-n 1000} for $1,000$ iterations. The input file is specified by the \texttt{-i <inputFile>} option and has to be in the format described in Section~\ref{sec:input}. A median of all the trees from \texttt{<inputFile>} is computed and the result is stored in \texttt{<outputFile>}. Typical usage may be:
\begin{verbatim}
 java -jar trap.jar --median -n 1000 -i myTrees -o myMedian
\end{verbatim}
which will compute an approximation of a median of the trees from the file \texttt{myTrees} using $1,000$ iterations and store the resulting tree in the file \texttt{myMedian}.


\subsection{Computing means}
To compute a median, choose the \texttt{--mean} option. Then you may require the cyclic-order version of the algorithm by selecting \texttt{-c}, see Section~\ref{sec:cycvran}. The number of iterations is a mandatory parameter, so type e.g. \texttt{-n 1000} for $1,000$ iterations. The input file is specified by the \texttt{-i <inputFile>} option and has to be in the format described in Section~\ref{sec:input}. A median of all the trees from \texttt{<inputFile>} is computed and the result is stored in \texttt{<outputFile>}. Typical usage may be:
\begin{verbatim}
 java -jar trap.jar --mean -n 1000 -i myTrees -o myMean
\end{verbatim}
which will compute an approximation of the mean of the trees from the file \texttt{myTrees} using $1,000$ iterations and store the resulting tree in the file \texttt{myMean}.


%%%%%%%%%%%%%%%%%%%%%%%%%%%%%%%%%%%%%%%%%%%%%%%%%%%%%%%%%%%%%%%%%%%%%%%%%%%%%%%%%%%%%%%%%%%%%%%%%%%%%%%%%%%%%%%%%%%%%%%%%%%%%%%%%

%%%%%%%%%%%%%%%%%%%%%%%%%%%%%%%%%%%%%%%%%%%%%%%%%%%%%%%%%%%%%%%%%%%%%%%%%%%%%%%%%%%%%%%%%%%%%%%%%%%%%%%%%%%%%%%%%%%%%%%%%%%%%%%%%
\section{Input} \label{sec:input}
The input file has to contain (at least two) trees in the Newick format, one tree per line. The leaves of each tree are labeled by integers $0,1,\dots,n.$ The leaf $0$ can represent the root, or the outgroup of the phylogenetic tree in question. An example of a tree in the Newick format is
\begin{verbatim}
 ((1:0.7,2:0.42):0.11,3:0.9,0:45);
\end{verbatim}


%%%%%%%%%%%%%%%%%%%%%%%%%%%%%%%%%%%%%%%%%%%%%%%%%%%%%%%%%%%%%%%%%%%%%%%%%%%%%%%%%%%%%%%%%%%%%%%%%%%%%%%%%%%%%%%%%%%%%%%%%%%%%%%%%

%%%%%%%%%%%%%%%%%%%%%%%%%%%%%%%%%%%%%%%%%%%%%%%%%%%%%%%%%%%%%%%%%%%%%%%%%%%%%%%%%%%%%%%%%%%%%%%%%%%%%%%%%%%%%%%%%%%%%%%%%%%%%%%%%
\section{Log file}
The computation details (e.g. run time) are displayed on the screen. If you want a copy to come as a text file, use the \texttt{-l} option.

%%%%%%%%%%%%%%%%%%%%%%%%%%%%%%%%%%%%%%%%%%%%%%%%%%%%%%%%%%%%%%%%%%%%%%%%%%%%%%%%%%%%%%%%%%%%%%%%%%%%%%%%%%%%%%%%%%%%%%%%%%%%%%%%%

%%%%%%%%%%%%%%%%%%%%%%%%%%%%%%%%%%%%%%%%%%%%%%%%%%%%%%%%%%%%%%%%%%%%%%%%%%%%%%%%%%%%%%%%%%%%%%%%%%%%%%%%%%%%%%%%%%%%%%%%%%%%%%%%%
\section{Cyclic versus random in median/mean computations} \label{sec:cycvran}
For both median and mean computations there are two versions of the algorithms: cyclic-order version and random-order version, see~\cite{mm}. Both versions converge to the same value, so most users have no preference which one to use. The random-order version is implemented as default, to choose the cyclic-order version instead, use the \texttt{-c} option. Notice that if the \texttt{-c} option is used, then the number of iterations specified by the \texttt{-n} option means the number of \emph{cycles,} that is, the actual number of iterations is then equal to the number of cycles times the number of trees.

%%%%%%%%%%%%%%%%%%%%%%%%%%%%%%%%%%%%%%%%%%%%%%%%%%%%%%%%%%%%%%%%%%%%%%%%%%%%%%%%%%%%%%%%%%%%%%%%%%%%%%%%%%%%%%%%%%%%%%%%%%%%%%%%%

%%%%%%%%%%%%%%%%%%%%%%%%%%%%%%%%%%%%%%%%%%%%%%%%%%%%%%%%%%%%%%%%%%%%%%%%%%%%%%%%%%%%%%%%%%%%%%%%%%%%%%%%%%%%%%%%%%%%%%%%%%%%%%%%%
% TO DO \section{Examples}

%%%%%%%%%%%%%%%%%%%%%%%%%%%%%%%%%%%%%%%%%%%%%%%%%%%%%%%%%%%%%%%%%%%%%%%%%%%%%%%%%%%%%%%%%%%%%%%%%%%%%%%%%%%%%%%%%%%%%%%%%%%%%%%%%

%%%%%%%%%%%%%%%%%%%%%%%%%%%%%%%%%%%%%%%%%%%%%%%%%%%%%%%%%%%%%%%%%%%%%%%%%%%%%%%%%%%%%%%%%%%%%%%%%%%%%%%%%%%%%%%%%%%%%%%%%%%%%%%%%
\section{Licence}
Copyright (C) 2013 Miroslav Bacak, Vojtech Juranek

This program is free software: you can redistribute it and/or modify it under the terms of the GNU General Public License as published by the Free Software Foundation, either version 3 of the License, or (at your option) any later version.

This program is distributed in the hope that it will be useful, but WITHOUT ANY WARRANTY; without even the implied warranty of MERCHANTABILITY or FITNESS FOR A PARTICULAR PURPOSE.  See the GNU General Public License for more details.

You should have received a copy of the GNU General Public License along with this program.  If not, see http://www.gnu.org/licenses/.

%%%%%%%%%%%%%%%%%%%%%%%%%%%%%%%%%%%%%%%%%%%%%%%%%%%%%%%%%%%%%%%%%%%%%%%%%%%%%%%%%%%%%%%%%%%%%%%%%%%%%%%%%%%%%%%%%%%%%%%%%%%%%%%%%

%%%%%%%%%%%%%%%%%%%%%%%%%%%%%%%%%%%%%%%%%%%%%%%%%%%%%%%%%%%%%%%%%%%%%%%%%%%%%%%%%%%%%%%%%%%%%%%%%%%%%%%%%%%%%%%%%%%%%%%%%%%%%%%%%
\bibliographystyle{siam}
\bibliography{manual}
\end{document}          
